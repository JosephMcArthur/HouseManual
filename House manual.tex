\documentclass[11pt]{article}
\usepackage[a4paper, portrait, margin=0.6in]{geometry}


\title{\textbf{Welcome to The Old Store}}
\author{Steve Smith \& Eleanor Raffan}
\begin{document}

\maketitle

\newpage

\section{Basics}
\subsection*{Contact details}
Address: The Old Store, High Street, Brinkley, Cambridgeshire CB8 0SE
\\Telephone no: 01638 507 280
\\Steve email: steve.smith.cam@gmail.com
\\Ellie email: er311@cam.ac.uk
\\Mobile while away: 07779 166820 (Steve), 07971802346 (Ellie) Rick Raffan (Ellie's Dad): +
\\UK contact while we are away: Ellie's Mum = Mrs Jenny Raffan.  Lives in Devon but knows the house well and would be a good person to speak to.  01647 252273  Doesn't do mobiles!  
\\If no luck there, you could try Steve's parents Paul and Tricia Smith who live in Barcelona, best by email.  Paul underscore smith underscore bcn@yahoo.es  (this silly program doesn't seem to like the underscore button!)  But avoid if possible as they have a lot on at the moment.

\subsection*{Travel Plans}

Date: Mon 9th Apr 18 : London Heathrow to Doha Intl , Qatar (DOH) Departing: 21:55, Mon; Arriving: 06:40, Flight No: Qatar Airways (QR) 16
Doha Intl , Qatar to Canberra Intl, Australia (CBR); Date: Tue 10 Apr 18 Departing: 09:20, Arriving: 09:00, Wed; Flight No: Qatar Airways (QR) 906

Date: Mon 30 Apr 18: Canberra Intl, Australia to Doha Intl , Qatar (DOH) Departing: 12:35, Mon; Arriving: 23:30, Mon; Flight No: Qatar Airways (QR) 907
Doha Intl , Qatar to London Heathrow, UK (LHR) Date: Tue 01 May 18 Departing: 01:40, Tue; Arriving: 07:05, Tue; Flight No: Qatar Airways (QR) 9

\subsection*{Security} 
The front door has a main latch which opens with a key from outside, and a deadlock which can be turned on the inside and unlocked from outside. A spare key is kept in the key safe on the wall near the floor to the right of the front door as you face it. The code is 1206. Please leave a set of keys in there at all times if you're not in.  Ali opposite has a key too in emergency and if she ends up doing the cat.  
\subsection*{Emergency Info}
Fuse box: this is in the cupboard above the fridge, you need to move the fridge out into the kitchen to get at it, so hopefully you won't need to! The mains water stopcock is in the cupboard in the bathroom in case of emergency...  There are torches in the cupboard at the top of the stairs, and are candles in the drawer of the large cupboard near the front door and a great gas lantern in the upstairs cupboard.

The cupboard at the top of the stairs is also a good place to find other general stuff you might need like gaffer tape or bike repair kits or fuses. 

Doctor - Orchard House in Newmarket. 

\subsection*{Neighbours}
Opposite at Norwood Cottage is Ali and her daughter Sophie.  Sophie (15yo) usually feeds the cat when we are away, for £4/day.  Knock on the door when Ali's car is there or ring.  Ali works from home so is often there.  Ali's mobile no is 07780014144.
\\Sylvia (house on right as you look at the road) is friendly but may have hedge issues so don't rely on her...
\\Roger and Hiliary (house on left as you look at the road) are lovely too.  
\\
\\If you need help or a second opinion on something about the house, I would suggest asking Ali in the first instance.  

\newpage 
\section{Cat}

Officially he's called Loki, but we tend to call him "cat", "kitten", "puss" or "aaaargh, grrrrr" depending on his current activities. He's an outdoor cat primarily, and doesn't use a litter tray. (He does know about litter trays and there is a gray tray under kitchen surface and litter on the 'up and over behind you' shelf in cupboard at the top of the stairs, but it's never been necessary except around vet visits.)  His cat flap out to the back is to the right of the oven, and is microchipped so only he can use it. 

He spends lots of the night out, and much of the day inside asleep on whichever surface is most appealing. We keep bedroom doors shut else he comes and sits on our head in the middle of the night, but it's not a problem if you want to let him into the bedroom.  He will sometimes miaow when he first hears you stir, or even before, in the hope of company and food.  If you ignore him, he will stop after a few minutes.  Be warned - if you let him in, he takes it as an indication you are a soft touch and has in the past yowled earlier and earlier at one particularly gullible lodger!
\subsection*{Food}
His food regime is a pouch of wet food and a scoop of biscuits three times daily (morning and, supper time and bedtime usually). IF he's eaten everything and is asking for more he can have an extra ad-hoc scoop of biscuits.  Conversely, if you are finding there's food in the bowls at mealtimes, either don't give more or throw old away and replace with fresh.  Would be good to avoid him becoming even more portly than he already has done this winter! 

We have had an ant problem recently - they have a good nose for finding uneaten cat food.  Yuk!  For that reason we have been feeding him with bowl on a box to raise it and are particularly careful about not dropping bits.  If you get ants, there are bait stations to put down (the ants eat the poison and take it home to kill the next, so this option is theoretically a more permanent solution) or spray (if you are grossed out and want a quick fix) in the cupboard at the top of the stairs.  Whatever, please make sure uneaten food is thrown away at least every 24 hours.  

We don't usually leave a water bowl as he prefers to drink from window condensation, washing up bowl, rainwater etc but put one out if you like.  Cat bowls live in the cupboard next to the wet food.  We use the dishwasher to clean them.    

He does hunt and occasionally leaves entire or remnants of mice or birds - sorry! If he's leaving them every day, he's probably getting too much food, so cut down on the feeding in that case! We call this 'titrating food to effect'. 

There is some wet food on the top shelf of the lower half of the kitchen cupboard and biscuits on the shelf below. More pouches in the box on bottom shelf and more biscuits in the bag under the island.

\subsection*{Veterinary care}
He is fully insured with PetPlan, which means that money isn't really an issue regarding vet care (which is not the same as saying we would do anything rather than put him to sleep, but does take money out of the equation).  He is registered at \\Cambridge Veterinary Group (http://www.cambridgevetgroup.co.uk/): \\89a Cherry Hinton Road, Cambridge, Cambridgeshire CB1 7BS
\\Phone: 01223 249331
\\Emergencies: 0845 500 4247 (out of office hours)

We have a very good vet friend called Karen who should be the first contact for any cat health related queries that aren't immediate emergencies, and first call after the emergency number if it is an emergency as she would know what kind of decisions we would be likely to make. Her details are:
\\Karen Humm, 07967 732 396

Should you need it, his travel basket is in the cupboard at the top of the stairs on the shelf up and over.

He is regularly flea treated.  
\newpage
\section{Living Here}
\subsection*{Fire}
There is a wood-burning stove beside the sofa, but we are very low on logs at this end of the winter. Feel free to use the fire - hopefully you won't need it, but if it's cold the fire is very effective. The lower section has the ash-tray for the fire, and a pair of levers to control air flow (both together = maximum flow, both apart = minimum, toggle the left hand lever first). I think you can get logs from Ocado. You might want fire lighters too, depending on boy scout credentials.

The night storage heaters are surprisingly effective. We will leave them as we have been using them and you can alter as you feel is necessary.  It is fine to turn them up (more input) but please make sure you avoid the red rocker on the heater as that is the (v expensive) immediate option as opposed to the more economical night storage option.  

Please avoid using the electric panel heater if you can because they are fiendishly expensive. If you are cold at night, consider ripping the electric blanket from our bed - it is marvellous!

\subsection*{Outside}
Garden a work in progress.  Garden chairs in cellar under dining table with a table too. Picnic rugs in cupboard by front door if ever sunny enough to go outside.  BBQ under the cover. Charcoal in cellar I think.  

\subsection*{Water}
Hot takes time to reach tap from tank. Runs out towards the end or running a long bath. You can put on constant or change timer (then change back) to heat extra.


\subsection*{Kitchen}
Dishwasher program pot symbol usually, salt recently used up so will need more if light goes on, tablets and rinseaid in cupboard L sink.

Washing up rules! DO NOT put the following in the dishwasher: Silver cutlery and bone handled knives, wooden plates, cast iron frying pans (but casserole dishes which are enamelled can go in), wooden plates or spatulas.  We've worked hard to try to 'cure' the cast iron frying pans - don't use metal scourer, just soak if needed and wipe with soapy water.  Don't soak wooden plates etc or bone handled knives as they can split.  The blue handled cutlery is fine in the dishwasher.  

Breadmaker is great and easy to use - instructions and recipes in the manual which is usually to left microwave.  Vitamin C is the white powder in small Kilner jar and yeast which is in the drawer lower left drawer, ditto milk powder.

Baking stuff in top green cupboard and the one under the hifi.  Larger dishes, jugs etc in sideboard and that cupboard too.

We tend to use oven on the setting which is at about 5 o'clock on left hand dial.  You need to set that one for source of heat and the left hand dial for temp.  Grill setting is at 3 o'clock and also needs temperature on. 

Bins under sink.  We use the square ish clear plastic lidded pots for compost as we've had problems with ants previously.  See elsewhere re bins out.  

Help yourselves to spices etc.  Similarly, feel free to use our dishwasher tablets/laundry liquid etc but please replace things you finish to a similar-ish value of what you used.
\subsection{Bedrooms and related} 
Towels and spare linen in cupboard in hall upstairs.  Blue/cream flowers for the large duvet in front bedroom, others fit double in spare room.  Summer weight duvets for both beds in cupboards around the place if you need them.  

Blind in spare room does go up but need to roll in right direction. 

We've done our best to clear storage space in bedrooms but apologies there isn't more, particularly for whoever gets the back bedroom. That person might want to hang clothes in the nursery.   

 
\subsection*{Bathroom/Laundry}
There is a spare bath mat on the washing machine so you can cycle between the two so as to keep the floor dry.  If the wooden side to the bath or wall get very wet, please dry them with a towel once you're out.  

Washing machine: Manual usually on shelf in bathroom, fairly self explanatory.  Rack for drying on landing, or use dryer.  

Dryer: empty water tank top left every couple of dryer cycles and empty filter (which you access with door open) at the same time.  Dryer not completely reliable in sensor drying so please either use set time (eg 60 mins) or if using sensor to stop make sure you are in the house and paying attention.  

Radio charges with power cable from any of the other radios.  

Small loo flush isn't very powerful.  Don't clog with too much loo roll and if you get into trouble, repeat flush or chuck a bucket of water down.  IMPORTANTLY, flush button sometimes gets stuck down and will keep emptying water so listen for persistent trickle of water and jiggle it up if required.  


\subsection*{Cleaning}
Mop, bucket, vacuum cleaner in cupboard at top of stairs. Empty vacuum cleaner cylinder into black bins outside when full.  Refils of washing up liquid, laundry liquid and general purpose detergent in bathroom cupboard (along with most other cleaning products).  
\subsection*{Bins}
The bins in the kitchen are under the sink. Please separate your rubbish into compostable waste (food etc, including meat scraps), recycling (metal, plastic bottles, card/paper etc) and general waste (mostly plastic film or plasticated card). Compostable goes in plastic lidded pots as above, recycling in the blue bin (no plastic bags) and general waste into white bin liners in the black bin. New liners/bags should be in the cupboard down and left of the sink, bottom shelf. 

When the bins under the sink get full, please empty them into the big bins outside. The black bin needs a black bin liner (there is usually a roll of them out there) whilst the recycling goes directly into the blue or green wheelie bin (no plastic bags in either please). The bins get taken away on a Thursday, with black (general) going every week and the blue/green recycling alternating weeks. On 	{Wednesday evening} please put the black bin liners (out of the bin) and whichever of the recycling bins needs to go onto the pavement outside. Check what the neighbours have put out and copy them!
\subsection*{Phone}
Mobile reception better at front of house.  Landline and a limited no of mobile calls free - help yourself.  Please don't spend too long on phone to mobiles though.  We use WhatsApp calls quite a lot to mobiles. 

Please check the answermachine if dialling tone is alternating and note messages - let us know if you think urgent.  

\section{Entertainment}
\subsection*{TV}
The TV is connected up to a BT box, and there are 2 TV remotes, one for the TV itself and the other for the receiver. The BT box is plugged into HDMI2, accessible via the "Source" button (top right on TV remote). We have Netflix which you're welcome to use - hit the blue button in the middle of the BT remote control and navigate to "Players and Apps" - "Players" - "Netflix". The rest should be pretty self-explanatory!  We don't have a subscription to other than Freeview and Netflix.
\subsection*{Computer}
We have unlimited broadband, which isn't lightning quick but should be ok - feel free to use it (probably best not do any pirating etc though!). The wifi password is on a small white card in the flip-down section of the cupboard under the TV. The study has a wireless keyboard and a monitor which you can connect laptops to (via HDMI or SVGA cables on the desk) as well as a scanner/printer via usb - feel free to use any of those.
\newpage
\section{Other things you need to know}
Bins on Wednesdays.

Our cleaner usually comes for 3h on Tuesdays and will continue to do so.  Anita is lovely, Hungarian, pretty good English.  We will give her some 'deep clean' jobs to do while we are away on the grounds that you won't make nearly as much mess as we usually do.  We pay her online and will continue to do so, so don't feel you need to offer her money.  

Very occasionally the kitchen sink backs up.  It's worse if you wash grease and food down, so don't.  If it happens, try plunging with plunger in cupboard upstairs, or get acid/drainbuster from www and use that.  We can repay.  Avoid plumber if possible by looking for signs it is draining slowly in advance of blockage and using those tricks early.  

Please can you keep an eye on our cars which will be parked on the lane up to the church +/- in drive?  White and black Seat, either or both will be here depending on what we decide re airport transport.  Keys by front door if you needed it for any reason, but please don't use it.  

Ordanance Survey map of the area is on shelf outside spare room for planning walks.  Lovely footpaths/byways to Weston Colville and Carlton and there is a stud just across the fields at Willingham Green where there are young thoroughbred foals in the fields in mornings/early afternoons.  Nice pubs in our village, Borough Green (you can walk there across the fields) and many other surrounding villages.  There is a bus service to Newmarket I think as well as Cambridge, though train quicker for the latter.  Newmarket has the racing museum and National Stud which does tours with more foals.  A morning trip to Newmarket is worth it one day just to experience being in a traffic jam caused by strings of racehorses!

If you want to borrow my bike (under shed at end garden) help yourself.  Key in bowl by front door on red lanyard.

Be careful of the mirror which is yet to be hung in bathroom.


\newpage
\section{Housesitting Ettiquette}
We are really pleased that you are going to be here while we are away, mainly as you will be keeping our cat company. We also benefit from you checking the house is ok and keeping it looking lived in.  In return, we hope you will enjoy living here for free and hope you will feel at home.  However, it is a bit worrying handing our house over to relative strangers and we would ask you to live by the following house rules.

\textbf{Respect our stuff and try not to damage anything.}  Please be particularly careful about ceramics and wooden stuff, much of which is made by Ellie's dad.  If things do get broken, please let us know and don't try to do a repair - leave that to us.  

\textbf{Do use our consumables but please replace things you finish and don't abuse it.}  ie feel free to use spices/detergents/loo roll and so on, but please keep a tally in your head and replace with equivalent value of something that gets used up and don't leave us without essentials.  Please leave the quick cook pasta in the fridge that we plan to eat when we get back, but use up any other food that will go out of date (some veg, dairy etc).  Similarly, please switch off lights, don't overuse heating, follow instructions re phone.  

\textbf{Clean up after yourselves.} Keep on top of the cleaning and at the end of your stay, please leave clean sheets on the bed in our bedroom (don't worry about the spare room).  Ditto, launder your sheets/towels and leave to dry.  Do a general house clean to a similar standard to that which you found on arrival.  

\textbf{If you move it, put it back.}  Goes for furniture, books or other stuff.  

\textbf{Keep in contact.} Please let us know if you have any worries about the house, or questions, or if you are going to be away overnight.  A brief 'hullo' email once a week would put our minds at rest.  If you find yourselves with a house emergency, of course you can act on your initiative but ring Ellie's mum or ask a neighbour for a second opinion before doing anything that might be expensive. 

\textbf{Keep all our mail and let us know if there is anything that looks particularly important.}  Ditto for answerphone.  

\textbf{If there's no compelling reason to touch our belongings, don't.}  Respect our trust, please.   

\textbf{No overnight guests without asking, except partners.}  No parties, please, but you are welcome to entertain small groups of friends so long as you take responsibility for them adhering to these rules. 

\textbf{Try to get along with the house and with each other and let us know if there are problems so we can help work things out.}  You both seem great, have good reasons to be here for the whole month and we hope we have provided a good home for you for that time.  But it is possible you might not get along or might encounter problems with the house that mean you don't want to stay.  Please, please let us know if you forsee a problem and we will do what we can to make things better.  

\textbf{The pussy cat is your priority!}  He loves company so it is great you will be here.  Don't encourage him to go anywhere near the road.  Be careful not to let him fool you into feeding him more than he needs - we have already let him get a bit fat this winter so don't let him balloon!  If you are both going to be away at any point, please arrange well in advance for Sophie to feed him twice a day.  You can contact her by knocking on the door at Norwood Cottage and speaking to her and her mum Ali, or text Ali on 07780014144

\section{Thank you very much :)} 

\end{document}
